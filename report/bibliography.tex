\documentclass {fkpset}

\lhead{Forest Kobayashi; Matthew LeMay}
\chead{Bibliography}
\rhead{Math 181 -- Spring, 2020}

\begin{document}
	
	\begin{center}
		\vspace{-2.0cm}
		\scshape \LARGE The Dynamics of Turing Machines
		\vspace{1.05cm}
	\end{center}
	
	
	\begin{thebibliography}{widestlabel}
		\bibitem{Siegelmann}
		Siegelmann, Hava T., and Shmuel Fishman. ``Analog computation with dynamical systems.'' \textit{Physica-Section D} 120.1 (1998): 214-235.
		
		This paper presents a way to understand the behavior of continuous dynamical systems as a form of computation. It is interesting because it talks about dynamical systems as models of computation, and analyzes them using tools from computational complexity.
	
		\bibitem{Moore}
		Moore, Cristopher. ``Unpredictability and undecidability in dynamical systems.'' 
		\textit{Physical Review Letters} 64.20 (1990): 2354.
		
		This paper analyzes the decidability of questions about three-dimensional dynamical systems. It is interesting because it examines the complexity of questions we might want to ask about dynamical systems.
		
		\bibitem{Kurka}
		K{\.u}rka, Petr. ``On topological dynamics of Turing machines.'' \textit{Theoretical Computer Science} 174.1-2 (1997): 203-216.
		
		This paper defines two dynamical systems based on the motion of the Turing Machine head and analyzes their behavior. It is interesting because it analyzes Turing Machines as dynamical systems, just like the Siegelmann paper analyzes dynamical systems as models of computation. 
		
		
		
	\end{thebibliography}
	
\end{document}