\documentclass{standalone}
\usepackage{tikz}
\usetikzlibrary{calc}


\usepackage{fkmath}
% Turing machine commands and stuff
\newcommand{\acceptState}{{\rm Accept}}
\newcommand{\rejectState}{{\rm Reject}}
\newcommand{\blank}{\textvisiblespace}

\newcommand{\acc}{\cmark}
\newcommand{\rej}{\cmark}

\newcommand{\lnext}{\circ}
\newcommand{\ltil}{\mathcal{U}}
\newcommand{\psc}{\mathcal{P}}





\begin{document}
  \begin{tikzpicture}

    % Define cell height (sqs == `square side')
    \pgfmathsetmacro{\sqs}{.8}

    % mincell gives the x coordinate of the leftmost cell, basically
    \pgfmathsetmacro{\mincell}{2}

    % Number of tape cells to draw total
    \pgfmathsetmacro{\numcells}{10}

    % How far to the right to go
    \pgfmathsetmacro{\maxcell}{\numcells - \mincell}

    % Define how far left/right the diagram should go so that we can
    \pgfmathsetmacro{\xmax}{\sqs * (\maxcell + 2)}
    \pgfmathsetmacro{\xmin}{\sqs * -1*(\mincell + 3)}

    % Draw tape extending outwards
    \draw (\xmin,0) -- (\xmax, 0);
    \draw (\xmin,\sqs) -- (\xmax, \sqs);


    % Let's actually assume we only write right for now
    % \pgfmathsetmacro{\min}{\sqs * (\numcells + 3)}


    % Draw the cells of the work tape
    \foreach \x in {-\mincell, ..., \maxcell}{
      % Left corner coordinates (we choose bottom left corner)
      \pgfmathsetmacro{\lcx}{\sqs * (\x-1)} % Correct an off-by-one error
      \pgfmathsetmacro{\lcy}{0}

      % Right corner coordinates (we choose top right corner)
      \pgfmathsetmacro{\rcx}{\sqs * \x}
      \pgfmathsetmacro{\rcy}{\sqs}

      \draw (\lcx, \lcy) rectangle (\rcx, \rcy);
    }

    % Write the struck-out version of \sigma_1
    \node (sg1) at (\sqs*.5, \sqs*.5) {$\color{blue} \gamma_i$};

    % Write the input string, sans sigma_1
    \foreach \x in {2, ..., 5}{
      \pgfmathsetmacro{\labelx}{\sqs*(\x -.5)}
      \pgfmathsetmacro{\labely}{\sqs*.5}
      \node (sg\x) at (\labelx, \labely) {$\sigma_{\x}$};
    }

    % Write the blank characters
    \foreach \x in {6, ..., \maxcell}{
      \pgfmathsetmacro{\labelx}{\sqs*(\x -.5)}
      \pgfmathsetmacro{\labely}{\sqs*.35}
      \node (sg\x) at (\labelx, \labely) {$\epsilon$};
    }

    \foreach \x in {-\mincell, ..., 0}{
      \pgfmathsetmacro{\labelx}{\sqs*(\x -.5)}
      \pgfmathsetmacro{\labely}{\sqs*.35}
      \node (sg\x) at (\labelx, \labely) {$\epsilon$};
    }


    % We wanna draw horizontal dots to indicate the tape continues; we
    % calculate the coordinates here
    % Make the coordinates for the left one
    \pgfmathsetmacro{\hdlx}{\sqs*(\maxcell+1)}
    \pgfmathsetmacro{\hdly}{\sqs*.5}

    % Make the coordinates for the right one
    \pgfmathsetmacro{\hdrx}{\sqs*(\mincell-6)}
    \pgfmathsetmacro{\hdry}{\sqs*.5}

    % Draw the horizontal dots extending to the right
    \node (rhdots) at (\hdlx, \hdly) {\huge $\cdots$};
    \node (lhdots) at (\hdrx, \hdry) {\huge $\cdots$};


    % Draw the previous head
    \pgfmathsetmacro{\tmphx}{\sqs*.5}
    \pgfmathsetmacro{\tmphy}{\sqs*(-1.5)}
    \node[circle, draw=black, dotted, opacity=.5] (headprev) at (\tmphx, \tmphy) {$q_0$};
    \draw[-latex, dotted, opacity=.2] ($(headprev) + (0, 0.5)$) -- ($(\tmphx, 0) + (0, -.1)$);


    % Turing machine head
    \pgfmathsetmacro{\tmhx}{\sqs*1.5}
    \pgfmathsetmacro{\tmhy}{\sqs*(-1.5)}
    \node[circle, draw=blue] (head) at (\tmhx, \tmhy) {$\color{blue} q_i$};
    \draw[-latex, blue] ($(head) + (0, 0.5)$) -- ($(\tmhx, 0) + (0, -.1)$);

    % Annotate the transition
    \draw[->] (headprev) to[bend right=80] (head);
    \node[below] () at (\sqs, -2.25*\sqs) {$R$};

  \end{tikzpicture}
\end{document}
